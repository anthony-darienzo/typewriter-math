\documentclass{./typewriter-math}
\usepackage[margin=1in]{geometry}

%\titlecolor{teal}
\title{MATH 500}
\subtitle{Problem Set 10}
\author{Anthony D'Arienzo}
%\contact{apd6@illinois.edu}
\date{4 April 2022}

\DeclareMathOperator\Ann{Ann}
\DeclareMathOperator\rank{rank}
\newcommand\tors{\text{tors}}

\begin{document}
	
	\maketitle%

	\begin{exercise}[(DF 12.1.6)]
		Let \(R\) be an integral domain and \(M\) any non-principal ideal of \(R\)
		such that \(M \neq \pwrap{0}\), considered as an \(R\)-module. Show that
		(i) \(M\) is torsion free and (ii) \(\rank M = 1\), but (iii) \(M\) is not
		a free module.

		\begin{proof}[Proof of (i)]
			Let \(x\) be an element of \(M\). Suppose \(r x = 0\), where \(r \neq
			0\). Since \(R\) is an integral domain, this implies \(x = 0\) (\(M\) is
			an ideal of \(R\), so \(x\) is an element of \(R\)). Thus, if \(x \in M\)
			is torsion, then \(x = 0\): \(M_\tors = \bwrap{0}\).
		\end{proof}

		\begin{proof}[Proof of (ii)]
			Let \(x,y\) be any two elements of \(M\). Then \(yx\) and \(xy\) are
			elements of \(M\). Integral domains are commutative, so \(yx - xy = yx -
			yx = 0\), so \(x\) and \(y\) are linearly dependent. Since \(x\) and
			\(y\) are arbitrary, this shows that \(\rank M < 2\). Since \(M \neq
			\pwrap{0}\), \(\rank M > 0\), so \(\rank M = 1\).
		\end{proof}

		\begin{proof}[Proof of (iii)]
			Suppose, for the purpose of deriving a contradiction, that \(M\) were a
			free module. Since the basis of a free module is always linearly
			independent, and \(\rank M = 1\), the basis of \(M\) must be a single
			element \(x_0 \in M\). That is, every element \(y\) of \(M\) is uniquely
			equal to an expression of the form \(r x_0\) for some \(r \in R\). \(M\)
			is an ideal of \(R\), so this implies \(M = \pwrap{x_0}\). Since \(M\) is
			not principal, we have a contradiction. Therefore \(M\) is not free.
		\end{proof}

	\end{exercise}

	\begin{exercise}[(DF 12.1.7)]
		Let \(R\) be any ring, \(A_1,\ldots,A_m\) be \(R\)-modules, and \(B_k
		\subseteq A_k\) submodules. Show that there is an isomorphism \(\pwrap{A_1 \oplus
		\ldots \oplus A_m}/\pwrap{B_1 \oplus \ldots \oplus B_m} \cong
		\pwrap{A_1/B_1} \oplus \ldots \oplus \pwrap{A_m/B_m}\) of modules.

		\begin{proof}
			Let \(M \eqdef \pwrap{A_1 \oplus \ldots \oplus A_m}/\pwrap{B_1 \oplus
			\ldots \oplus B_m}\) and \(N \eqdef \pwrap{A_1/B_1} \oplus \ldots \oplus
			\pwrap{A_m/B_m}\). Then \(N\) is the coproduct of the modules \(A_i/B_i\).
			Therefore it suffices to show that \(M\) satisfies the universal property
			of this coproduct. Suppose we have a family of morphishms \(f_i: A_i/B_i
			\to X\) for \(i \in \bwrap{1,\ldots,m}\), where \(X\) is some
			\(R\)-module. Let \(\pi_i: A_i \to A_i/B_i\) be the quotient morphism.
			Then \(f_i \circ \pi_i : A_i \to X\) is a morphism of \(R\)-modules. The
			universal property of \(A_1 \oplus \ldots \oplus A_m\) tells us that there
			exists a unique morphism \(\hat{f}: A_1 \oplus \ldots \oplus A_m \to X\)
			such that the following diagram commutes for all \(i \in
			\bwrap{1,\ldots,m}\).%
			\begin{equation}\label{eq:f-hat}%
				\begin{tikzcd}
					A_i \arrow[r,"\pi_i"] \arrow[dr,"\iota_i"] & A_i/B_i \arrow[r,"f_i"] &
					X\\
					& A_1 \oplus \ldots \oplus A_m \arrow[ur,"\hat{f}",dashed,swap] & 
				\end{tikzcd}
			\end{equation}%
			where \(\iota_i: A_i \to A_1 \oplus \ldots \oplus A_m\) is the canonical
			inclusion morphism.  Furthermore \(\iota_i(B_i)\) is contained in the
			kernel of the map \(A_i \to A_1 \oplus \ldots \oplus A_m \to M\) since
			\(\iota_i(B_i)\) is a submodule of \(B_1 \oplus \ldots \oplus B_m\).
			Therefore the universal property of the quotient gives us a unique map
			\(\rho_i: A_i/B_i \to M\) fitting into the following diagram.%
			\begin{equation}\label{eq:rho-i}%
				\begin{tikzcd}
					A_i \arrow[r,"\pi_i"] \arrow[dr,"\iota_i",swap] & A_i/B_i
					\arrow[dd,"\exists !  \rho_i",dashed, bend left=100]\\
					& A_1 \oplus \ldots \oplus A_m \arrow[d] \\
					& M
				\end{tikzcd}
			\end{equation}%
			Lastly, we show that \(B_1 \oplus \ldots \oplus B_m\) is contained in the
			kernel of \(\hat{f}\). Note that it suffices to show that \(\iota_i(B_i)\)
			is contained in the kernel of \(\hat{f}\), because (by definition) \(B_1
			\oplus \ldots \oplus B_m = \iota_1(B_1) \oplus \ldots \oplus
			\iota_m(B_m)\). Since \(B_i\) is contained in \(\ker \pi_i \circ f_i\),
			Diagram~\ref{eq:f-hat} shows that \(\iota_i(B_i)\) is contained in \(\ker
			\hat{f}\), as desired. Therefore \(B_1 \oplus \ldots \oplus B_m \leq \ker
			\hat{f}\); hence \(\hat{f}\) itself induces a unique morphism \(\tilde{f}:
			M \to X\) making the following diagram commute.%
			\begin{equation}\label{eq:f-tilde}%
				\begin{tikzcd}
					A_i \arrow[r,"\pi_i"] \arrow[dr,"\iota_i",swap] & A_i/B_i
					\arrow[r,"f_i"] & X\\
					& A_1 \oplus \ldots \oplus A_m \arrow[ur,"\hat{f}",swap] \arrow[d] & \\
					& M \arrow[uur,"\exists ! \tilde{f}",dashed,swap, bend right=30] &
				\end{tikzcd}
			\end{equation}%
			Using Diagram~\ref{eq:rho-i}, we can replace the left side with
			\(\rho_i\):%
			\[%
				\begin{tikzcd}
					A_i/B_i \arrow[d,"\rho_i",swap] \arrow[r,"f_i"] & X\\
					M \arrow[ur,"\exists ! \tilde{f}", dashed, swap] &
				\end{tikzcd}
			\]%
			This is the universal property for the coproduct \(N\). Therefore \(M
			\cong N\).
		\end{proof}
	\end{exercise}

	\begin{exercise}[(DF 12.1.8)]
		Let \(R\) be a PID, \(B\) a torsion \(R\)-module, and \(p\) a prime element
		in \(R\). Prove that if \(pb = 0\) for some \(b \in B, b \neq 0\), then
		\(\Ann(B) \subseteq \pwrap{p}\). (Recall that \(\Ann(B) \eqdef \bwrap{ r
		\in R \mid r B = 0 }\)).

		\begin{proof}
			\(R b\) is a submodule of \(B\), so it is a module in its own right. Since
			\(p b = 0\), \(p \in \Ann(R b)\); hence \(\Ann(R b) \neq \pwrap{0}\).
			Since \(b \neq 0\), \(1 \cdot b \neq 0\), so \(\Ann(R b)\) is not the unit
			ideal. Therefore \(\Ann(R b)\) is a nontrivial ideal in the PID \(R\), so
			\(\Ann(R b)\) is a principal ideal. Since \(p\) is prime, \(\pwrap{p}\) is
			maximal amoung principal ideals, therefore \(\Ann(R b) = \pwrap{p}\). Any
			element \(r \in R\) which annihilates all of \(B\) will annihilate \(b\)
			in particular. Thus \(\Ann(B) \subseteq \Ann(R b)\); hence \(\Ann(B)
			\subseteq \pwrap{p}\).
		\end{proof}
	\end{exercise}

	\begin{exercise}
		Let \(R\) be a ring with \(1\), not necessarily commutative. Show that if
		\(I,J \subseteq R\) are left ideals such that \(J = Ia\) for some \(a \in
		R^\times\), then \(R/I \cong R/J\) as modules.

		\begin{proof}
			We will show that \(R/I\) satisfies the universal property of \(R/J\).
			Let \(\phi: R \to N\) be a morphism of \(R\)-modules such that \(J
			\subseteq \ker \phi\). We define a new morphism \(\overline{\phi}: R/I
			\to N\) of \(R\)-modules by setting:%
			\[%
				\overline{\phi}\pwrap{\overline{x}} \eqdef \phi\pwrap{x a}.
			\]%
			To show that this is well-defined, suppose \(y \in I\), so there exists
			an element \(z \in J\) such that \(y = z a^{-1}\). Then \(y a) = \phi(z)
			= 0\) since \(z \in \ker \phi\); hence \(\overline{\phi}\) is a
			well-defined (set)-function from \(R/I\) to \(N\). We now show that
			\(\overline{\phi}\) is linear:%
			\[%
				\overline{\phi}\pwrap{r \overline{x} + s \overline{y}} = \phi\pwrap{
				\pwrap{ r x + s y } a } = \phi\pwrap{ r x a + s y a } = r\phi(x a) + s
				\phi(y a) = r \overline{\phi}(\overline{x}) + s
				\overline{\phi}(\overline{y}).
			\]%
			Furthermore we can define an \(R\)-module morphism \(\pi': R \to R/I\) by
			setting \(\pi'(x) \eqdef \overline{x a^{-1}}\). Since \(J = I a\), \(\ker
			\pi'\) contains \(J\) (in fact, \(J = I a\) implies \(\ker \pi' = J\)).
			Moreover, \(\overline{\phi}\pwrap{\pi'(x)} = \phi(x a^{-1} a) =
			\phi(x)\), so the following triangle commutes.%
			\[%
				\begin{tikzcd}%
					R \arrow[r,"\phi"] \arrow[d,"\pi'",swap] & N\\
					R/I \arrow[ur,"\overline{\phi}",dotted,swap] &
				\end{tikzcd}
			\]%
			All that is left to check is that \(\overline{\phi}\) is unique. Since
			\(\pi'\) is surjective, the requirement that \(\overline{\phi} \circ \pi'
			= \phi\) forces a unique definition for \(\overline{\phi}\).

			Thus \(\pi': R \to R/I\) satisfies the universal property of the quotient
			\(R/J\). Since quotients are unique up to isomorphism, \(R/I \cong R/J\).
		\end{proof}
	\end{exercise}

	\begin{exercise}
		Let \(R \eqdef M_{2\times 2}(F)\) with \(F\) a field. Find examples of
		left ideals \(I,J \subseteq R\) such that \(R/I \cong R/J\) as modules but
		\(I \neq J\).

		\begin{proof}
			Let \(I\) denote the collection of all matrices \(A \in R\) such that \(A
			\pwrap{\begin{smallmatrix} 1 \\ 0 \end{smallmatrix}} = 0\), and let \(J\)
			denote the collection of all matrices \(B \in R\) such that
			\(B\pwrap{\begin{smallmatrix} 0 \\ 1 \end{smallmatrix}} = 0\).

			\(I\) is a left-ideal: for any \(M_1, M_2 \in R\) and \(A_1,
			A_2 \in I\),%
			\[%
				\pwrap{M_1 A_1 - M_2 A_2}\begin{pmatrix} 1 \\ 0 \end{pmatrix} = M_1
				\pwrap{ A_1 \begin{pmatrix}1\\0\end{pmatrix} } - M_2 \pwrap{ A_2
				\begin{pmatrix}1\\0\end{pmatrix} } = 0.
			\]%
			A similar argument shows that \(J\) is also a left-ideal. Furthermore,
			the matrix%
			\[%
				\sigma \eqdef \begin{pmatrix}
					0 & 1 \\
					1 & 0
				\end{pmatrix}
			\]%
			is an element of \(R^\times\) (it is its own inverse). Since \(I \sigma =
			J\), Exercise~\ref{ex:4} implies \(R/I \cong R/J\). However, \(I \neq
			J\): \(\pwrap{\begin{smallmatrix} 0 & 0 \\ 0 & 1 \end{smallmatrix}}\) is
			an element of \(I\) but not an element of \(J\).
		\end{proof}
	\end{exercise}

	\begin{exercise}[(DF 12.2.10)]
		Find all similarity classes of \(6 \times 6\) matrices over \(\Q\) with
		minimal polynomial \(\pwrap{x+2}^2\pwrap{x-1}\).

		\begin{warning_box*}
			It suffices to give all lists of invariant factors and write out some of
			their corresponding matrices.
		\end{warning_box*}

		\begin{proof}
			Let \(P \eqdef \pwrap{x+2}^2\pwrap{x-1}\). Let \(V \eqdef \Q^6\) and \(T:
			V \to V\) be a matrix with minimal polynomial \(P\). Then \(V_T\)
			decomposes into a sum%
			\[%
				V_T \cong \Q[x]/\pwrap{f_1} \oplus \ldots \oplus \Q[x]/\pwrap{f_k},
			\]%
			where \(f_1 \mid f_2 \mid \ldots \mid f_k\) are all monic polynomials.
			Since the minimal polynomial is the largest invariant factor, \(f_k = P\).
			Furthermore \(\dim_F V_T = \deg f_1 + \ldots + \deg f_k\). \(\deg f_k =
			3\), so \(\deg f_1 + \ldots + \deg f_{k-1} = 3\), leaving the following
			possibilities: \begin{itemize}%
				\item \(k = 4\) and \(\deg f_1 = \deg f_2 = \deg f_3 = 1\),
				\item \(k = 3\) and \(\deg f_1 = 1\) and \(\deg f_2 = 2\),
				\item \(k = 2\) and \(\deg f_1 = 3\).
			\end{itemize}%
			Since the invariant factors must divide \(P\), this yields six
			classes.\begin{enumerate}%
				\item \(f_1 = f_2 = P\);
				\item \(f_1 = f_2 = f_3 = x-1\) and \(f_4 = P\);
				\item \(f_1 = f_2 = f_3 = x+2\) and \(f_4 = P\);
				\item \(f_1 = x+2\), \(f_2 = \pwrap{x+2}^2\), and \(f_3 = P\);
				\item \(f_1 = x+2\), \(f_2 = \pwrap{x+2}\pwrap{x-1}\), and \(f_3 = P\);
				\item \(f_1 = x-1\) and \(f_2 = \pwrap{x+2}\pwrap{x-1}\), and \(f_3 = P\).
			\end{enumerate}%
			Any matrix is similar to its rational canonical form, so these six classes
			correspond bijectively to the collection of similarity classes of
			\(M_{6\times 6}(\Q)\). The invariant factors determine the rational
			canonical form of the representative of each similarity class. In the
			order enumerated above, the following six matrices represent the six
			similarity classes.%
			\[%
				\begin{array}{ccc}%
					\begin{pmatrix}
						\begin{matrix}
							0 & 0 & 4\\
							1 & 0 & 0\\
							0 & 1 & -3
						\end{matrix} & \\
						& \begin{matrix}
							0 & 0 & 4\\
							1 & 0 & 0\\
							0 & 1 & -3
						\end{matrix}
					\end{pmatrix},&
					\begin{pmatrix}
						\begin{matrix}
							1 &  & \\
							 & 1 & \\
							 &  & 1
						\end{matrix} & \\
						& \begin{matrix}
							0 & 0 & 4\\
							1 & 0 & 0\\
							0 & 1 & -3
						\end{matrix}
					\end{pmatrix},&
					\begin{pmatrix}
						\begin{matrix}
							-2 &   &  \\
							  & -2 &  \\
							  &   & -2
						\end{matrix} & \\
						& \begin{matrix}
							0 & 0 & 4\\
							1 & 0 & 0\\
							0 & 1 & -3
						\end{matrix}
					\end{pmatrix},\\[0.5in]
					\begin{pmatrix}
						\begin{matrix}
							-2 &   &  \\
							  & 0 & -4\\
							  & 1 & -4
						\end{matrix} & \\
						& \begin{matrix}
							0 & 0 & 4\\
							1 & 0 & 0\\
							0 & 1 & -3
						\end{matrix}
					\end{pmatrix},&
					\begin{pmatrix}
						\begin{matrix}
							-2 &   &  \\
							  & 0 & 2\\
							  & 1 & -1
						\end{matrix} & \\
						& \begin{matrix}
							0 & 0 & 4\\
							1 & 0 & 0\\
							0 & 1 & -3
						\end{matrix}
					\end{pmatrix},&
					\begin{pmatrix}
						\begin{matrix}
							1 &   &  \\
							  & 0 & 2\\
							  & 1 & -1
						\end{matrix} & \\
						& \begin{matrix}
							0 & 0 & 4\\
							1 & 0 & 0\\
							0 & 1 & -3
						\end{matrix}
					\end{pmatrix}.
				\end{array}%
			\]%
		\end{proof}
	\end{exercise}

	\begin{exercise}[(DF 12.3.5)]
		Compute the Jordan canonical form (over \(\C\)) of the matrix%
		\[%
			A \eqdef \begin{pmatrix}
				1 & 0 & 0\\
				0 & 0 & -2\\
				0 & 1 & 3
			\end{pmatrix}.
		\]%
		\begin{proof}[Solution]
			The characteristic polynomial \(\det (\lambda I - A)\) is \(\lambda^2 - 3
			\lambda + 2 = \pwrap{\lambda - 2}\pwrap{\lambda - 1}\), so the eigenvalues
			of \(A\) are \(2\) and \(1\). We can directly solve the system \(A - 2 I =
			0\) and \(A - I = 0\) to find an eigenbasis%
			\[%
				\begin{array}{ccc}
					\begin{pmatrix} 1\\0\\0 \end{pmatrix},&
					\begin{pmatrix} 0\\-2\\0 \end{pmatrix},&
					\begin{pmatrix} 0\\1\\-1 \end{pmatrix}.
				\end{array}
			\]%
			Therefore these three vectors form a basis of \(V_A\); hence the Jordan
			canonical form of \(A\) is%
			\[%
				J(A) = \begin{pmatrix}
					1 & 0 & 0\\
					0 & 1 & 0\\
					0 & 0 & 2
				\end{pmatrix}.
			\]%
		\end{proof}
	\end{exercise}

	\begin{exercise}
		Let \(R \eqdef \Z\fb{i}\). Classify all finitely generated \(R\)-modules
		\(M\) with the property that \(5M = 0\).

		\begin{proof}
			\(R\) is a PID, so due to the elementary divisor composition, any finitely
			generated \(R\)-module \(M\) is isomorphic to a direct sum of the form:%
			\[%
				M \cong R/\pwrap{p_1^{k_1}} \oplus \ldots \oplus R/\pwrap{p_m^{k_m}},
			\]%
			where each \(p_i\) is a prime element of \(R\). This decomposition is
			unique in the sense that \(m\) is unique and the \(p_i^{k_i}\) are unique
			up to reordering and multiplication by units. Therefore it suffices to
			classify \(R\)-modules \(M'\) of the form \(R/\pwrap{p^k}\), where \(p\)
			is a prime element of \(R\) such that \(5 M' = 0\). Since \(R\) is a PID,
			\(\Ann(M') = \pwrap{r_0}\) for some \(r_0 \in R\). Since \(\pwrap{p^k}\)
			is an ideal, \(\pwrap{p^k}\) is contained in \(\Ann(M')\), so \(r_0\)
			divides \(p^k\). That is, \(r_0\) is a power of \(p\). On the other hand
			\(5 M' = 0\), so \(5 \in \Ann(M') = \pwrap{r_0}\). This shows that \(r_0\)
			is a \emph{nontrivial} power of \(p\). Furthermore, \(5 =
			\pwrap{2+i}\pwrap{2-i}\); both of these factors are prime (they are
			irreducible in a PID). So we have \(\pwrap{2+i}\pwrap{2-i} = p^d a\),
			where \(0 < d \leq k\) and \(a \in R\). Thus \(d = 1\) and \(p = 2+i\) or
			\(p = 2-i\). In other words, \(M' \cong R/\pwrap{2+i}\) or \(M' \cong
			R/\pwrap{2-i}\).
			
			Returning to the direct sum \(M\), this tells us that \(M\) is a direct
			sum of copies of \(M_1 \eqdef R/\pwrap{2+i}\) and \(M_2 \eqdef
			R/\pwrap{2-i}\). This proves the following classification.

			\begin{proposition}
				For any finitely generated \(\Z[i]\)-module \(M\) such that \(5 M = 0\),
				there exist \(d_1,d_2 \in \Z_{\geq 0}\)	such that%
				\[%
					M \cong M_1^{d_1} \oplus M_2^{d_2}.
				\]%
			\end{proposition}
		\end{proof}
	\end{exercise}
\end{document}
